\documentclass[11pt,a4paper]{scrartcl}
\usepackage[utf8]{inputenc}
\usepackage[T1]{fontenc}
\usepackage[ngerman]{babel}
\usepackage[top=2.5cm, bottom=3cm]{geometry}
\usepackage{scrpage2}

%\usepackage{cite}

\pagestyle{scrheadings}
\clearscrheadfoot
\ihead{\today \\
M. Herhold, C. Pfeiffer, N. Kortenbruck, J. Dietz, R. Mueller, M. Rudolph
}  
\ohead{hg17b}
\cfoot{\pagemark}

%now following: lots of stuff for sorted lists from https://tex.stackexchange.com/questions/121489/alphabetically-display-the-items-in-itemize
\usepackage{datatool}% http://ctan.org/pkg/datatool
\newcommand{\sortitem}[2][\relax]{%
  \DTLnewrow{list}% Create a new entry
  \ifx#1\relax
    \DTLnewdbentry{list}{sortlabel}{#2}% Add entry sortlabel (no optional argument)
  \else
    \DTLnewdbentry{list}{sortlabel}{#1}% Add entry sortlabel (optional argument)
  \fi%
  \DTLnewdbentry{list}{description}{#2}% Add entry description
}
\newenvironment{sortedlist}{%
  \DTLifdbexists{list}{\DTLcleardb{list}}{\DTLnewdb{list}}% Create new/discard old list
}{%
  \DTLsort{sortlabel}{list}% Sort list
  \begin{itemize}%
%Das \l=sortlabel in der nächsten und das [\l] in der Übernächsten wurden von Christopher Pfeiffer hinzugefügt damit die labels angezeigt werden.
    \DTLforeach*{list}{\theDesc=description, \l=sortlabel}{%
      \item[\l] \theDesc}% Print each item
  \end{itemize}%
}


\begin{document}
\subject{\vspace{-2cm}}
\title{\vspace{-0.5em}Qualitätssicherungskonzept}
\subtitle{\vspace{1ex}hg17b - Android App für Weiterbildungsmanagement}
\author{M. Herhold, C. Pfeiffer, N. Kortenbruck, J. Dietz, R. Mueller, M. Rudolf\vspace{-0.5em}}%autor
\date{\vspace{-0.5em}\today}
\maketitle

\tableofcontents
\newpage

\section{Dokumentationskonzept}
Eine gute Dokumentation des Projektes hilft nicht nur unserer Gruppe beim verwirklichen des Projektes, sondern auch eventuellen zukünftigen Programmierern, die unsere App weiterentwickeln wollen. Die ausführliche Dokumentation erleichtert uns die Zusammenarbeit, da wir uns besser in den Quellcode der anderen einarbeiten können. \newline
Im allgemeinen ist in unserer Gruppe jeder Programmierer für die Dokumentation seines Quellcodes selbst verantwortlich. Sämtliche Dokumentation und Namensvergabe im Quellcode erfolgt auf Englisch.

\subsection{Coding Standard}
Die Nutzung eines Coding Standards verbessert die Lesbarkeit und führt zu schnellerem Verständnis des Quellcodes. Wir nutzen die Java Code Conventions\footnote{http://www.oracle.com/technetwork/java/codeconventions-150003.pdf}, da wir hauptsächlich mit Java programmieren. Auch hier ist jeder Programmierer selbst in der Verantwortung sich an den Standard zu halten.

\subsection{Interne Dokumentation}
Kommentare im Quellcode sollen genutzt werden um eine Übersicht über den Code zu geben und zusätzliche Informationen zu liefern, welche nicht direkt aus dem Quellcode ersichtlich sind. Die Informationen müssen relevant sein um das Programm zu lesen und zu verstehen. Desweiteren soll doppelte Information vermieden werden, d.h. selbstbeschreibende Funktionen müssen nicht kommentiert werden.

\subsection{Quelltextnahe strukturierte Dokumentation}
Wir nutzen das Javadoc Tool um die quelltextnahe Dokumentation als HTML Dokument zur Verfügung zu stellen. Deswegen versehen wir unsere Klassen und Methoden mit Kommentaren der Form "/** Kommentar */" 

\section{Testkonzept}
Um am Ende ein gutes Produkt abzuliefern, müssen wir unser Programm ausgiebig auf Fehler überprüfen. Deshalb testen wir sowohl die Klassen und Methoden als einzelnes, als auch das Zusammenspiel der Komponenten. 
\subsection{Komponententests}
Um Fehler so früh wie möglich zu erkennen werden Komponententests durchgeführt. Dabei wird überprüft ob Methoden und Klassen korrekt funktionieren. Das heißt zum einen das diese sowohl das gewünschte Ergebnis liefern, als auch das vermeiden von Fehlern bei allen möglichen Eingaben.\newline
Für unsere Komponententests verwenden wir das Framework JUnit, welches speziell zum testen von Java Komponenten entwickelt wurde. Jeder Programmierer testet seine Komponenten selbst.
\subsection{Integrationstests}
Die Integrationstests dienen dazu die korrekte Zusammenarbeit voneinander abhängiger Komponenten zu testen. Erfolgreich getestete Komponenten werden hier zusammengebracht. Im Schwerpunkt der Integrationstests stehen die Schnittstellen der einzelnen Komponenten. Um den Testaufwand nicht unnötig steigen zu lassen, betrachten wir Subsysteme mit bestandenen Integrationstests wieder als einzelne Komponenten und verknüpfen diese mit anderen abhängigen Komponenten mit erneuten Integrationstests bis das gesamte Programm verknüpft ist.
\subsection{Systemtests}
Im Systemtest wird die Software auf einer Umgebung geprüft die der endgültigen Anwendung möglichst nahe kommt. In diesem Schritt werden die im Lastenheft gestellten funktionalen und nichtfunktionalen Anforderungen an das Produkt geprüft. Die hier genutzen Testdaten sollen den tatsächlichen Daten des Projektes möglichst nahe sein.
\subsection{Abnahmetest}
Im letzten Test wird Das Produkt dem Auftraggeber vorgeführt und mit dessen Vorstellungen abgeglichen. Ist der Kunde zufrieden wird das Produkt übergeben und das Projekt ist abgeschlossen.
\section{Organisationskonzept}
\subsection{Gruppentreffen}
Das Team trifft sich jeden Donnerstag 9:00 Uhr. Bei diesen Treffen können sich die Teammitglieder über den aktuellen Stand des Projektes austauschen, die nächste Abgabe planen und entsprechende Aufgaben verteilen, sowie offene Fragen mit den anderen Gruppenmitgliedern oder gegebenenfalls dem Auftraggeber oder Betreuer zu klären. Diese Treffen werden für nicht anwesende Gruppenmitglieder protokolliert.\newline 
Zusätzliche Treffen finden nach Absprache meist Montags 11:00 Uhr am Tag einer Abgabe statt.
\subsection{Kommunikation}
Um kleinere Fragen an das gesamte Team oder an einzelne Mitglieder stellen zu können nutzen wir zur zusätzlichen Kommunikation Slack. Fragen spezifisch zu bestimmten Issues werden im GitLab bei dem entsprechenden Issue als Kommentar vermerkt.
\end{document}