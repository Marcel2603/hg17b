\documentclass[a4paper,11pt]{article}
\usepackage{a4wide,german,url}
\usepackage[utf8]{inputenc}
\parindent0pt
\parskip3pt

\title{Thema „Android App für Weiterbildungsmanagement“\\ Detaillierter
  Themenaufriss }
\author{Hans-Gert Gräbe}
\date{Version vom 12.11.2017}

\begin{document}
\maketitle

Das Grundszenario geht von einer Integrationsplattform aus, über die
verschiedene Anbieter Veranstaltungen verwalten. Die Integrationsplattform
greift dabei auf verschiedene andere Plattformen (im Weiteren:
Primärplattformen) zu, welche die Informationen über Veranstaltungen und
Veranstalter verwalten und die relevanten Informationen im RDF-Format als
Webservice zur Verfügung stellen.

Auf eine davon unabhängige Weise soll eine Menge potenzieller Teilnehmer
erfasst werden. Die Veranstalter sollen in der Lage sein, während der
Veranstaltung zu registrieren, wer von den potenziellen Teilnehmern anwesend
ist.

Diese Informationen sollen angemessen öffentlich ausgewertet werden (welche
Veranstaltungen, wie viele Teilnehmer pro Veranstaltung, Highscore-Listen der
Teilnehmer nach verschiedenen Veranstaltungstypen usw.), wobei
datenschutzrechtliche Aspekte einen hohen Stellenwert haben.

Als besonderes Problem für die Modellierung ist zu berücksichtigen, dass die
verschiedenen Primärplattformen verschiedene Modelle für Events, Veranstalter,
Orte, Akteure usw. verwenden und auch die Datenqualität zu wünschen übrig
lässt.

Referenzmodell für die Datenmodellierung ist das Leipzig Data Event Modell, das
im Zuge der Arbeiten ggf. auf abgestimmte Weise modifiziert werden kann.

Da das Android-SDK eine Java-Plattform ist, soll auch das Backend in Java
entwickelt werden. Für die Anbindung der Datenschicht soll dazu das Apache Jena
Framework verwendet werden. Ggf. ist ein RDF Store und SPARQL Endpunkt unter
Verwendung einschlägiger Apache-Technologien aufzusetzen.

Die Integrationsinfrastruktur der Primärplattformen befindet sich aktuell im
Aufbau.  Einbezogen sind derzeit
\begin{itemize}
\item die Leipzig Data Event Plattform\\
  \url{http://leipzig-data.de/ontology/events/}, 
\item die Plattform „Leipziger Ecken“\\
  \url{http://pcai042.informatik.uni-leipzig.de/~graebe/le-rdf/}
\item und die Plattform „Nachhaltiges Leipzig“\\
  \url{http://pcai042.informatik.uni-leipzig.de/~graebe/nl-rdf/}.
\end{itemize}

\end{document}
