

\documentclass[11pt,a4paper]{scrartcl}
\usepackage[utf8]{inputenc}
\usepackage[T1]{fontenc}
\usepackage[ngerman]{babel}
\usepackage[top=2.5cm, bottom=3cm]{geometry}
\usepackage{scrpage2}
\usepackage{listings}
\usepackage{graphicx}
 \usepackage{color}
 \definecolor{middlegray}{rgb}{0.5,0.5,0.5}
 \definecolor{lightgray}{rgb}{0.8,0.8,0.8}
 \definecolor{orange}{rgb}{0.8,0.3,0.3}
 \definecolor{yac}{rgb}{0.6,0.6,0.1}

%\usepackage{cite}

\pagestyle{scrheadings}
\clearscrheadfoot
\ihead{\today \\
Julian Dietz, Marcel Herhold, Nikolai Kortenbruck
}  
\ohead{hg17b}
\cfoot{\pagemark}

%now following: lots of stuff for sorted lists from https://tex.stackexchange.com/questions/121489/alphabetically-display-the-items-in-itemize
\usepackage{datatool}% \usepackage{•} 
\newcommand{\sortitem}[2][\relax]{%
  \DTLnewrow{list}% Create a new entry
  \ifx#1\relax
    \DTLnewdbentry{list}{sortlabel}{#2}% Add entry sortlabel (no optional argument)
  \else
    \DTLnewdbentry{list}{sortlabel}{#1}% Add entry sortlabel (optional argument)
  \fi%
  \DTLnewdbentry{list}{description}{#2}% Add entry description
}
\newenvironment{sortedlist}{%
  \DTLifdbexists{list}{\DTLcleardb{list}}{\DTLnewdb{list}}% Create new/discard old list
}{%
  \DTLsort{sortlabel}{list}% Sort list
  \begin{description}%Im orginal Code itemize, aber sieht scheiße aus
  
%Das \l=sortlabel in der nächsten und das [\l] in der Übernächsten wurden von Christopher Pfeiffer hinzugefügt damit die labels angezeigt werden.
    \DTLforeach*{list}{\theDesc=description, \l=sortlabel}{%
      \item[\l] \theDesc}% Print each item
  \end{description}%
}



\begin{document}
\lstset{language=Java,
 basicstyle=\scriptsize\ttfamily,
   keywordstyle=\bfseries\ttfamily\color{red},
   stringstyle=\color{green}\ttfamily,
   commentstyle=\color{middlegray}\ttfamily,
   emph={square}, 
   emphstyle=\color{blue}\texttt,
   emph={[2]root,base},
   emphstyle={[2]\color{yac}\texttt},
   showstringspaces=false,
   flexiblecolumns=false,
   tabsize=2,
   numbers=left,
   numberstyle=\tiny,
   numberblanklines=true,
   stepnumber=1,
   numbersep=10pt,
   xleftmargin=15pt
}
\subject{\vspace{-2cm}}
\title{\vspace{-0.5em}Entwurfsbeschreibung\_release2}
\subtitle{\vspace{1ex}hg17b - Android App für Weiterbildungsmanagement}
\author{Julian Dietz, Marcel Herhold, Nikolai Kortenbruck\vspace{-0.5em}}%autor
\date{\vspace{-0.5em}\today}
\maketitle

\tableofcontents
\newpage

\section{Allgemeines}
Wir entwickeln eine Android-App für Schüler und Veranstalter von Weiterbildungsevents. Schüler bekommen eine Übersicht über ihre bereits besuchten Veranstaltungen und haben die Möglichkeit, kommende Veranstaltungen zu finden und sich unverbindlich anzumelden. Veranstalter können in der App die tatsächlich anwesenden Schüler, mithilfe einer dem Schüler zugeteilten ID, erfassen. 
\section{Produktübersicht}
Unser Produkt besteht aus einem Frontend und einem Backend. \newline
Das Frontend ist die Android App. Beim Start der App wird unterschieden ob der Nutzer ein Schüler oder ein Veranstalter ist. Schüler geben dabei ihre ID ein und erhalten Einblick in ihre Eventdaten, zukünftige Events sowie ihren Platz in der Bestenliste. Veranstalter können sich nach einem Klick auf die untere Schaltfläche au­to­ri­sie­ren und anschließend die Teilnehmer einer Veranstaltung erfassen. Das Frontend wurde mithilfe von Android Studio entwickelt.\newline
Unser Backend besteht aus einem Server mit RDF-Triplestore in welchem die Eventdaten und die Daten der Schüler gespeichert werden. Der Server nutzt hierfür das Apache-Jena Framework.
\section{Grundsätzliche Struktur- und Entwurfsprinzipien}
Die Daten der Schüler (IDs, Score, Rang) liegen zentral auf einem Server. Die Kommunikation zwischen mobiler Android App und Server laufen über eine auf dem Server liegende Java Applikation. Diese stellt einen Server Socket/SSL Server Socket zur Verfügung über den ein Client Socket/SSL Client Socket der App mit dem Server kommuniziert.
\section{Struktur- und Entwurfsprinzipien einzelner Pakete}
\subsection{Server}
Die (Schüler-) Daten auf dem Server liegen im SPARQL-Format als Triples in einem Graphen vor und werden von der PupilDB Java-Klasse aus abgerufen und manupuliert. Diese definiert die dafür benötigten SPARQL- Query und Update Statements. Die Server Klasse der auf dem Server laufenden Java Applikation handelt den Verbindungsaufbau und die Abfragen/Antworten vom/an den Client. Nachfolgend noch eine Auflistung der Klassen welche zum Server gehören 
\subsubsection{Server(Klasse)}
Diese Klasse ist die Main Klasse des Servers und erstellt die Datenbank und startet den Server. Meldet sich ein Client an so wird ein Objekt der Klasse Handler erstellt. Außerdem kann der Server eine Liste mit allen aktiven Clients anzeigen
\subsubsection{Commands}
Durch diese Klasse werden die Befehle für die Serveradministratoren zur Überwachung des Servers bereitgestellt. Im moment währen dies:
"list" (eine liste aller verbundenen Clients
"stop" zum herunterfahren des Servers
"help" um alle Befehle aufzulisten
\subsubsection{KeyHandler}
Der KeyHandler ist für das Key Management zuständig. Sie enthält den Server Trust Store und man kann keys abgleichen/prüfen ob diese im Key Store vorhanden sind.
\subsubsection{Handler}
Handler ist die Klasse welche sich um die Kommunikation zwischen Client, Server und Datenbank kümmert dabei wartet sie auf Anfragen des Clients und sendet Daten aus PupilDB und wird beendet wenn der Client offline ist. In der Handler Klasse findet außerdem serverseitig der Abgleich/das Empfangen der Keys für die Veranstalter Seite statt. Für den empfangenen, noch nicht aktivierten key wird außerdem ein Aktivierungslink erzeugt.
\subsubsection{KeyAdder}
Die eigenständige, ausführbare KeyAdder.java wird beim Aufrufen des Aktivierungslinks aufgerufen. Ein übertragener Key wird in den Trust Store des Servers geschrieben.
\subsubsection{PupilDB}
Die Klasse PupilDB stellt die Datenbank des Projekts und liefert Methoden zum manipulieren der Schüler sowie zum lesen der Schüler- und Veranstaltungsdaten. Die Schülerdaten werden dabei als  Triple Graphen gespeichert. Die Methoden nutzen SPARQL-Querys und Update Statements. 
\subsubsection{Mail}
Die Klasse Mail wird genutzt um automatisch E-Mails zu verschicken. Wenn ein Veranstalter sich zum ersten Mal in der App anmeldet, schickt der Server ihm über diese Klasse eine E-Mail mit einem Bestätigungslink. Dieser Link aktiviert das ssh Schlüsselpaar.
\subsection{Applikation}
Die Kommunikation mit dem Server wird in der Client Klasse der App gehandelt. In der StartActivity Klasse wird das Start Verhalten und der Login der Nutzer definiert. Die Klassen Logout, NextEvents, Ranking und Settings (zu finden in app/src/main/java/hg17b) sind jeweils für die Methoden der in der App vorhandenen Seiten Logout, Nächste Events, Ranking und Einstellungen zuständig. Die Layouts der Seiten werden in den dementsprechenden fragment xml Dateien definiert (zu finden in app/src/main/res/layout). Das Layout des Hauptmenüs hingegen wird in der navigation\_\textbackslash menu unter /\textbackslash verb /App/app/src/main/res/menu definiert. Hier noch eine Auflistung der Klassen der App
\subsubsection{Client}
Die Klasse Client erstellt einen Hintergrundthread welcher sich mit dem Server verbindet. Sie sendet und empfängt Daten von der Datenbank des Servers. In ihr werden außerdem keys an den Server gesendet wenn sich ein neuer Veranstalter anmeldet.
\subsubsection{KeyHandler}
Der Key Handler handelt den KeyStore der App. Es können neue Keys erzeugt werden.
\subsubsection{StartActivity}
Diese Klasse ist die Startseite der App und definiert beim starten das Layout und erstellt eine Instanz der Klasse Client. Nach dem klicken auf "Bestätigen" wird die eingegebene ID mit der Datenbank abgeglichen und wenn diese vorhanden ist wird zur Klasse Main weitergeleitet. Von hier aus kommt man auch zum Organizer login 
\subsubsection{Main}
Die Klasse Main bestimmt das Design unserer Hauptseite. Das ist die Seite welche ein Schüler nach dem Einloggen als erstes sieht. Oben wird die eigene ID angezeigt, in der Mitte erfährt der Schüler wie viele Punkte er hat und unten werden bevorstehende Veranstaltungen angezeigt, für welche der Schüler sich angemeldet hat
\subsubsection{Nextevents}
Die Klasse NextEvents zeigt die bevorstehenden Veranstaltungen an. Dazu nutzt sie die Eventdatenbank, welche seit dem 2. Release integriert ist.
\subsubsection{Lastevents}
Die Klasse LastEvents zeigt die vom eingeloggten Schüler besuchten Veranstaltungen an. Auch diese Klasse nutzt dazu die Eventdatenbank.
\subsubsection{Ranking}
Die Klasse zeigt die Topliste, also die höchsten Punktzahlen in absteigender Reihenfolge, geliefert aus der Klasse PupilDB. Unter der Rangliste wird die Platzierung des angemeldeten Schülers gezeigt.
\subsubsection{Setting}
In dieser Klasse sollen dem Schüler Einstellungsmöglichkeiten zur Individualisierung der App geboten werden. Zurzeit ist die implementierung dieser Funktionen ein Kann-Ziel vom 2. Release.
\subsubsection{Logout}
Die Klasse LogOut dient zum trennen der Serververbindung, klickt man auf den Button abmelden, so wird der Hintegrundthread geschlossen und man landet auf der StartActivity Seite um sich erneut anzumelden.
\subsubsection{OrganizerLogin}
Mit dieser Klasse können sich Veranstalter anmelden. dazu benötigen sie eine verifizierte E-Mail Adresse. Mit klick auf bestätigen wird das Menü für Veranstalter initialisiert.
\subsubsection{OrganizerMain}
Die Klasse OrganizerMain bestimmt das Layout der Hauptseite für Veranstalter. Nach dem einloggen ist dies die erste Seite die ein Veranstalter sieht. Außerdem werden bevorstehende Veranstaltungen angezeigt.
\subsubsection{OrganizerNextevents}
Hier werdem dem eingeloggten Veranstalter kommende Events angezeigt.
\subsubsection{OrganizerLastevents}
In dieser Klasse kann ein angemeldeter Veranstalter Informationen zu seinen vergangenen Veranstaltungen einsehen.
\newpage
\section{Datenmodell}
\fbox{ \includegraphics[width=1.1\textwidth]{Datenmodell}}
\section{Glossar}
\begin{sortedlist}
%Liste wird autoatisch alphabetisch nach dem Inhalt von [] sortiert.
\sortitem[ID]{(Identifikationsnummer) Im Zusammenhang dieses Projektes ist die Identifikationsnummer auf den Pässen der Schüler gemeint.}
\sortitem[RDF] {RDF steht für Resource Description Framework. Es wurde von der W3C als Standard zur Beschreibung von Metadaten konzipiert. Es gilt als ein grundlegender Baustein des Semantischen Webs. RDF ähnelt den klassischen Methoden zur Modellierung von Konzepten wie UML-Klassendiagramme und ER-Modell.}
\sortitem[Frontend] {Frontend beschreibt das auf dem Endgerät des Nutzers laufende Programm.}
\sortitem[Backend] {Backende beschreibt das auf dem Server laufende Programm, ein SPARQL Endpunkt (Apache Jena), sowie Java und Java Applikationen zum bearbeiten diverser Anfragen.}
\sortitem[Schüler] {Der Schüler nimmt am Ferienpass teil. Er soll die App leicht bedienen können.}
\sortitem[Veranstalter] {Der Veranstalter organisiert Freizeitaktivitäten für die Schüler. Er soll mit der App den Code des Schülers einscannen.}
\end{sortedlist}
\end{document}
