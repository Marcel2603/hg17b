\documentclass[11pt,a4paper]{scrartcl}
\usepackage[utf8]{inputenc}
\usepackage[T1]{fontenc}
\usepackage[ngerman]{babel}
\usepackage[top=2.5cm, bottom=3cm]{geometry}
\usepackage{scrpage2}

\pagestyle{scrheadings}
\clearscrheadfoot
\ihead{\today \\
%author
}  
\ohead{hg17b}
\cfoot{\pagemark}

\begin{document}
\titlehead{hg17b}
\subject{\vspace{-2cm}Protokoll}
\title{\vspace{-0.5em}}%Anlass des Protokolls
\subtitle{hg17b}%Untertitel/ Thema
\author{\vspace{-0.5em}}%autor
\date{\vspace{-0.5em}\today}
\maketitle

\tableofcontents
\newpage

\section{Begriffe}
\paragraph{Android}
Android ist sowohl ein Betriebsystem als auch eine Software-Plattform für verschiedene mobile Geräte. Andriod wird als freie Software entwickelt (d.h. Quelloffen). Die Basis dafür ist der Linux-Kernel.

%https://de.wikipedia.org/wiki/Android_(Betriebssystem) 22.11


\paragraph{API}

Api steht für Application-Programming-Interface , im deutschen wird es Umgangssprachlich meist als Programmierschnittstelle bezeichnet. Dies wird im Web meist zum Austausch und der Weiterverarbeitung von Daten und Inhalten zwischen verschiedenen Webseiten , Programmen und Content-Anbietern. APIS dienen also in der Informatik der vereinheitlichen und strukturierten Datenübergabe zwischen Programmen und Programmteilen.


%https://www.gruenderszene.de/lexikon/begriffe/application-programming-interface-api 22.11

%http://www.omkt.de/api/ 22.11


\paragraph{SPARQL}

SPARQL ist eine graphenbasierte Abfragesprache für RDF Der Name ist ein rekursives Akronym für \textbf{S}PARQL \textbf{P}rotocol \textbf{A}nd \textbf{R}DF \textbf{Q}uery \textbf{L}anguage. Die RDF Date Access Working Group (DAWG) des WWW Consortium(W3C) . Seit dem 21.3.2013 ist die W3C Recommendation für SPARKQL 1.1 veröffentlicht worden.

%https://de.wikipedia.org/wiki/SPARQL 22.11


\paragraph{RDF}

RDF steht für Resource Description Framework. Es wurde anfang von der W3C als Standard zur Beschreibung von Metadaten Konzipiert. Mittlerweile gilt RDF als ein grundlegender Baustein des Semantischen Webs. RDF ähnelt den klassischen Methoden zur Modellierung von Konzepten wie UML-Klassendiagramme und ER-Modell.

%https://de.wikipedia.org/wiki/Resource_Description_Framework  22.11


\paragraph{QR Code}

QR Code steht für Quick Response Code und ist ein 2D-Code der von Handys, Smartphones und Tablets eingescannt und ausgelesen werden kann. In diesen Codes können Webadressen, Telefonnummern , SMS und freier Text untergebracht werden. Sie verbinden damit virtuelen und physische Welt und spielen im Marketing und Publikationswesen eine Rolle

%http://wirtschaftslexikon.gabler.de/Definition/qr-code.html 22.11

\paragraph{Datenbanksystem} 
% ( https://de.wikipedia.org/wiki/Datenbank )
Ein Datenbanksystem besteht aus Datenbank, dem Datenbestand (den tatsächlichen Daten, die auf einem Speichermedium abgelegt sind), und dem Datenbankmanagementsystem (DBMS), dass den Zugriff auf die Daten ermöglicht. Direkter Zugriff auf die Daten erfolgt dabei nur zu administrativen Zwecken, andere Interaktionen mit den Daten erfolgen ausschließlich über das DBMS.\\
Eine Datenbank ist ein System zur effizienten, widerspruchsfreien und dauerhaften elektronischen Speicherung von Daten. Die Datenbank stellt eine Datenbanksprache als Schnittstelle zur Verfügung. Über sie lassen sich Daten abfragen und manipulieren.\\
Das Datenbankmodell legt die Grundlage für die Beziehungen und Strukturierung der Daten fest. Neben der gebräuchlichsten relationalen Datenbank, in der Daten in Tabellen gespeichert vorliegen und Beziehungen zu anderen Daten über Tabelleneinträge bestimmt werden, gibt es auch andere Datenbankmodelle, wie das Objektorientierte, bei dem Objekte Eigenschaften von anderen Objekten erben, oder das hierarchische, in denen Objekte in einer Eltern-Kind-Beziehung zueinander stehen.\\
Das Datenbankmanagementsystem legt das Datenbankmodell fest. Es definiert die Logik mit der Daten abgefragt und gespeichert werden und interagiert mit der Datenbank über die Datenbanksprache.

\paragraph{Auszeichnungssprache}
% ( https://de.wikipedia.org/wiki/Auszeichnungssprache#Vereinfachte_Auszeichnungssprachen )
Eine Auszeichnungssprache ist eine maschinenlesbare Sprache für Gleidrung und Formatierung von Texten, wie beispielsweise HTML. Sie beschreibt Eigenschaften und Darstellungsformen ihres Inhaltes (in der Regel mit Tags).

\paragraph{XML}
% ( https://de.wikipedia.org/wiki/Extensible_Markup_Language https://en.wikipedia.org/wiki/XML )
Die Erweiterbare Auszeichnungssprache (Extensible Markup Language) XML ist eine Auszeichnungssprache zur Darstellung hierarchisch strukturierter Daten in Textform, die sowohl von Mensch als auch Maschine verstanden werden. Es eignet sich damit für den plattform- und implementationsunabhängigen Austausch von Daten zwischen unterschiedlichen Computersystemen.\\
Die Zeichen die ein XML Dokument enthält werden zwischen Textauszeichnung (Markup) und Inhalt unterschieden. Texauszeichnungen beginnen mit dem Zeichen < und Enden mit dem Zeichen >. Alle Zeichen, die nicht zwischen diesen stehen sind Inhalt.\\
Mit Hilfe dieser Unterscheidung sind Tag definiert, Konstrukte, die mit einem < beginnen und mit einem > enden. Man unterscheidet zwischen 3 unterschiedlichen Tags:
\begin{itemize}
\item Start Tag der Form <Tag1>
\item End Tags der Form </Tag1>
\item Leerer Tag, der Anfangs- und End-Tag zugleich ist und keinen Inhalt enthält, der Form 	<Tag1 />
\end{itemize}

Die wichtigste Struktureinheit eines XML Dokumentes ist das Element. Sie sind Träger der Informationen eines XML Dokumentes und können als Inhalt Text wie auch weitere Elemente haben. Alle Elemente mit Inhalt besitzen einen Anfangs- und End-Tag. Beispiel:\\
<Element1> Inhalt</Element1>\\
Einem Element können auch Attribute zugeordnet werden. Dies geschieht im Start Tag. Ein Attribut besteht aus einem Namen und einem Wert. Name und Wert werden im Start Tag definiert/zugeordnet, beispielsweise wird dem Element „Element1“ das Attribut „Anzahl“ mit dem Wert „3“ wie folgt zugeordnet:\\
<Element1 Anzahl=“3“> Inhalt</Element1>\\
XML Dokumenten kann eine Document Type Description (DTD) zugeordnet werden, die bestimmt welche Elemente ein XML Dokument enthalten darf. In Dokumenten ohne DTD kann der Name von Elementen frei gewählt werden. Liegt eine DTD zugrunde, so definiert sie welche Elemente ein XML Dokument besitzt. Beispielsweise ist XHTML eine solche DTD, die Elemente wie <html>, <head>, <body> aber auch Überschriften wie <h1> oder Pragraphen <p> als erlaubte Elemente des Dokumentes zu Grunde legt.

\paragraph{Markdown}
% ( https://de.wikipedia.org/wiki/Markdown )
Markdown ist eine vereinfachte Auszeichnungssprache, die mittels Konvertierungssoftware in XHTML umgewandelt werden kann. Im Gegensatz zu Tags in XHTML verwendet Markdown meist Satzzeichen für die Auszeichnung von Text, beispielsweise das Doppelkreuz für die Auszeichnung einer Überschrift.

\paragraph{Web Feed}
% ( https://de.wikipedia.org/wiki/Web-Feed )
Ist eine Technik zur Veröffentlichung von Änderungen auf Websites, bei der die Initiative zum Nachrichtenempfang vom Empfänger der Nachricht ausgeht.

\paragraph{RSS Feed}
% ( https://de.wikipedia.org/wiki/RSS_(Web-Feed) )
Ein RSS Feed ist ein Web Feed der Adressaten mit Daten im RSS Format zuführt, kurzen Informationsblöcken die meist aus Schlagzeile, Textanriss und Link zur Originalseite bestehen. RSS steht (aktuell) für Really Simple Syndication. Ein Abonnent muss dafür einen RSS-Channel abonnieren, eine spezielle Service Website, der RSS Client such in regelmäßigen Abständen nach Aktualisierungen. Informationsblöcke werden im XML Format übermittelt, das den Inhalt strukturiert übermittelt aber keine Informationen über Layout zur Verfügung stellt.

\paragraph{Java}
% ( https://de.wikipedia.org/wiki/Java_(Programmiersprache) )
Java ist eine objektorientierte Programmiersprache. Zum Ausführen von Java Quellcode wird eine Laufzeitumgebung (JRE – Java Runtime Environment) benötigt, die Bytecode in einer virtuellen Maschine ausführt. Der Code wird also nicht direkt auf der dem System zugrunde liegenden Maschine ausgeführt, sondern in einer virtuellen. Dies hat den Zweck, dass Java Quellcode Plattformunabhängig ist. Sie ist unabhängig von der Architektur des ausführenden Rechners, solange dort eine JRE läuft.

\paragraph{Framework}
Ein Framework ist ein Programmiergerüst, dass wiederverwendbare, gemeinsame Strukturen für die Entwicklung von Applikationen zur Verfügung stellt. Dies kann Support-Programme, Compiler, Code Bibliotheken, Werkzeugsätze und Schnittstellen (API) beinhalten. Es dient dazu, dass häufig verwendete Funktionen bereits enthalten sind und nicht immer neu programmiert werden müssen.

\section{Konzepte}
\paragraph{Semantic Web}
Die Bezeichnung Semantic Web wurde für das Web 3.0 kreiert. Beim semantischen Web handelt es sich um eine Web-Technologie bei dem die Suchmaschinen Informationen zueinander in Beziehung setzen, sie eigenständig auswerten und aus ihnen eine Bedeutung entnehmen können. Durch diesen Ansatz können Menschen und Computer wesentlich besser miteinander kooperieren und es können intelligentere Webservices kreiert werden. In einem solchen Web werden die Informationen in Beziehungen zueinander gestellt und verwaltet.
Während Menschen Informationen aus dem gegebenen Kontext schließen können und derartige Verknüpfungen unbewusst aufbauen, muss Maschinen dieser Kontext erst beigebracht werden; hierzu werden die Inhalte mit weiterführenden Informationen verknüpft. Semantic Web beschreibt dazu konzeptionell einen „Giant Global Graph“. Dabei werden sämtliche Dinge von Interesse identifiziert und mit einer eindeutigen Adresse versehen als Knoten angelegt, die wiederum durch Kanten (ebenfalls jeweils eindeutig benannt) miteinander verbunden sind. Zur Realisierung des Semantic Webs dienen Standards zur Veröffentlichung und Nutzung maschinenlesbarer Daten, u.a. sind dies:
\begin{itemize}
\item URIs in der doppelten Rolle zur Identifizierung von Entitäten und zum Verweisen auf weitergehende Daten dazu\\
\item RIF für die Darstellung von Regeln\\ 
\item RDF als gemeinsames Datenmodell zur Repräsentation von Aussagen\\
\item SPARQL als Anfragesprache und -protokoll\\ 
\item eine Reihe von verschiedenen Syntaxen um RDF-Graphen auszutauschen:
	\begin{itemize}
	\item RDF/XML, eine XML-Syntax. Lange Zeit die einzige standardisierte Syntax
	\item Turtle, eine Syntax die dem Tripelmodell näherkommt
	\item RDFa, um RDF in XML-Dokumenten, insbesondere XHTML, einzubetten …
	\end{itemize}
\\ 
\end{itemize}
%http://www.itwissen.info/Semantisches-Web-semantic-web.html 22.11.2017
%https://de.wikipedia.org/wiki/Semantic_Web 22.11.2017

\paragraph{Continuous Integration (CI)}
Kontinuierliche Integration (auch fortlaufende oder permanente Integration; englisch continuous integration) beschreibt den Prozess des fortlaufenden Zusammenfügens von Komponenten zu einer Anwendung. Das Ziel der kontinuierlichen Integration ist die Steigerung der Softwarequalität. Üblicherweise wird dafür nicht nur das Gesamtsystem neu gebaut, sondern es werden auch automatisierte Tests durchgeführt. Der gesamte Vorgang wird automatisch ausgelöst durch Einchecken in die Versionsverwaltung. Vorteile sind hierbei:
\begin{itemize}
\item Integrations-Probleme werden laufend entdeckt und behoben (gefixt) – nicht erst kurz vor einem Meilenstein\\
\item Frühe Warnungen bei nicht zusammenpassenden Bestandteilen, und sofortige Unit Tests entdecken Fehler zeitnah\\ 
\item Ständige Verfügbarkeit eines lauffähigen Standes für Demo-, Test- oder Vertriebszwecke\\
\item des Weiteren steht mit GitLab CI ein freier, in Ruby geschriebener Server für fortlaufende Integration zum Einsatz mit GitLab zu Verfügung.\\ 
\end{itemize}
Vergleiche hierzu außerdem Punkt 4.2 „Continuous Integration“ aus der Handreichung zum Softwaretechnik-Praktikum 2017/18.
%https://de.wikipedia.org/wiki/Kontinuierliche_Integration 22.11.2017

\paragraph{Android SDK / Android Studio}
Das Android Software Development Kit (kurz Android SDK) ist eine Entwicklungsumgebung und Sammlung von Tools für die Softwareentwicklung für das Android-Betriebssystem. In erster Linie wendet sich das SDK an Entwickler für die Erstellung von Android-Apps, allerdings hält es auch für einen Nutzer von Android nützliche Tools vor, wie bspw. Tools zum Flashen eines Recovery oder das Ansehen des Systemlogs.
Das Android SDK ist für Windows, Linux und Mac OS verfügbar und benötigt für viele der Hauptfunktionen ein JDK(Java SE Development Kit). Mittlerweile wurde das Android SDK in das neue Entwickler-Tool "Android Studio" integriert.  
%https://www.droidwiki.org/wiki/Android_SDK 22.11.2017
Android Studio ist eine freie Integrierte Entwicklungsumgebung (IDE) von Google und offizielle Entwicklungsumgebung für Android, der Quelltext von Android Studio ist frei verfügbar. 
Das Android Studio bündelt alle benötigten Funktionen für die Entwicklung und das Debugging von Apps. Außerdem enthält es unter anderem auch einen Layout-Editor, in dem Benutzeroberflächen  erstellt und für unterschiedliche Auflösungen getestet werden können. Es ist außerdem möglich, Google-Dienste wie Google Cloud Messaging innerhalb der IDE zu konfigurieren und direkt auf die App anzuwenden.   
%https://de.wikipedia.org/wiki/Android_Studio 22.11.2017

\section{Aspekte}
\paragraph{Django}
% ( https://de.wikipedia.org/wiki/Django_(Framework) )
Django ist ein in Python geschriebenes quelloffenes Webframework das einem Model-View-Presenter-Schema(MVC) folgt. Es gilt als Python-Gegenstück zu Ruby on Rails.

\paragraph{Gitlab}
% ( https://de.wikipedia.org/wiki/GitLab )
Gitlab ist eine Webanwendung zur Versionsverwaltung für Softwareprojekte auf Basis von git. Sie beinhaltet dieverse Mangement und Bug-Tracking-Tools. Gitlab wurd mit den Programmiersprachen Ruby und Go entwickelt.

\paragraph{Jekyll}
% ( http://magazin.phlow.de/jekyll/was-ist-jekyll/ )
Jekyll ist ein sogenannter Static Website Generator. Es besitzt keine eingene Oberfläche, sondern wird über die Kommandozeile ausgeführt. Jekyll baut statische HTML-Webseiten anhand von einfachen Textdateien im Dateiformat .md . Das Programm eignet sich sehr gut für den Betrieb eines Bloges oder für kleine Website-Projekte.

\paragraph{Ruby}
% ( https://www.ruby-lang.org/de/about/ )
Ruby ist eine objektorientierte Programmiersprache, die sich einfach anwenden und produktiv einsetzen lässt. Sie wurde Mitte der 90er entwickelt und entstand aus Perl, Smalltalk, Eiffel, Ada und Lisp. Sie hat eine elegante Syntax, die man leicht lesen und schreiben kann.

\paragraph{Scrum}
% ( Vorlesung SWT )
Scrum ist eine agile Methode, die allerdings das Management fokussiert und nicht spezielle agile Praktiken. Es wird in 3 Phasen untergliedert: Initialphase,Sprint-Zyklen, Projekt-Abschluss:\\
Initialphase: Genereller Überblick über das Zielsystem\\
Sprint-Zyklen: Entwicklung der einzelnen Inkremente des Systems\\
Projekt-Abschluss: Benötigte Dokumentation und Erfahrungsberichte\\ 



\end{document}