\documentclass[11pt,a4paper]{scrartcl}
\usepackage[utf8]{inputenc}
\usepackage[T1]{fontenc}
\usepackage[ngerman]{babel}
\usepackage[top=2.5cm, bottom=3cm]{geometry}
\usepackage{scrpage2}

%\usepackage{cite}

\pagestyle{scrheadings}
\clearscrheadfoot
\ihead{\today \\
M. Herhold, C. Pfeiffer, N. Kortenbruck, J. Dietz, R. Mueller, M. Rudolph
}  
\ohead{hg17b}
\cfoot{\pagemark}

%now following: lots of stuff for sorted lists from https://tex.stackexchange.com/questions/121489/alphabetically-display-the-items-in-itemize
\usepackage{datatool}% http://ctan.org/pkg/datatool
\newcommand{\sortitem}[2][\relax]{%
  \DTLnewrow{list}% Create a new entry
  \ifx#1\relax
    \DTLnewdbentry{list}{sortlabel}{#2}% Add entry sortlabel (no optional argument)
  \else
    \DTLnewdbentry{list}{sortlabel}{#1}% Add entry sortlabel (optional argument)
  \fi%
  \DTLnewdbentry{list}{description}{#2}% Add entry description
}
\newenvironment{sortedlist}{%
  \DTLifdbexists{list}{\DTLcleardb{list}}{\DTLnewdb{list}}% Create new/discard old list
}{%
  \DTLsort{sortlabel}{list}% Sort list
  \begin{description}%Im orginal Code itemize, aber sieht scheiße aus
  
%Das \l=sortlabel in der nächsten und das [\l] in der Übernächsten wurden von Christopher Pfeiffer hinzugefügt damit die labels angezeigt werden.
    \DTLforeach*{list}{\theDesc=description, \l=sortlabel}{%
      \item[\l] \theDesc}% Print each item
  \end{description}%
}


\begin{document}
\subject{\vspace{-2cm}}
\title{\vspace{-0.5em}Lastenheft}
\subtitle{\vspace{1ex}hg17b - Android App für Weiterbildungsmanagement}
\author{M. Herhold, C. Pfeiffer, N. Kortenbruck, J. Dietz, R. Mueller, M. Rudolph\vspace{-0.5em}}%autor
\date{\vspace{-0.5em}\today}
\maketitle

\tableofcontents
\newpage

\section{Ausgangssituation}
Innerhalb des nächsten Jahres sollen Schüler/innen von Leipziger Schulen  für Zukunftsthemen im außerunterrichtlichen Bereich sensibilisiert werden. Vorbild ist dabei das Trierer Zukunftsdiplom. Dabei sollen interessierte Schüler/innen einen Zukunftspass erhalten, der eine ID zur eindeutigen Identifizierung beinhaltet. Die Zuordnung zum Namen ist nur der jeweiligen Schule bekannt. Schüler die an einer gewissen Mindestanzahl an Veranstaltungen teilgenommen haben, sollen am Ende eine Auszeichnung bekommen. \\
Bisher gibt es bereits Ferienpässe, jedoch existiert im Moment noch keine digitale Struktur um eine schnelle Erfassung der Teilnehmer/innen einer Veranstaltung zu Realisieren. 
Eine solche soll nun rund um nachhaltiges-leipzig.de (welches Veranstalter und Events erfasst) erstellt werden.
\section{Zielsetzung und Produkteinsatz}

Im Rahmen des Projektes soll eine App entwickelt werden, über die sich die Schüler, durch Eingabe oder Scannen ihrer IDs, anschauen können an wie vielen Events sie teilgenommen haben. Außerdem soll eine Liste von weiteren Events einsehbar sein.\\
Gleichzeitig soll es eine Möglichkeit für Veranstalter geben sich bei dieser App anzumelden um somit Teilnehmer ihrer Events zu erfassen, damit im System eingetragen werden kann, dass die Person zu diesem Event anwesend war. Dazu sollen die IDs auf den Ausweisen der Schüler vom Veranstalter über die App gescannt werden können.\\ 
Des Weiteren soll die dazu nötige Infrastruktur entworfen und prototypisch implementiert werden. Die App soll als Android App, auf Basis des Android SDKs entwickelt, werden.

\section{Funktionale Anforderungen}

\subsection{Muss-Ziele}
\subsubsection*{Startbildschirm und Anmeldefunktion}
\begin{itemize}
%AF steht für Anmeldefunktion
\item[AF010] Der Benutzer wird beim erstmaligem Start der App oder wenn keine gespeicherten Anmeldedaten vorliegen aufgefordert sich zu identifizieren.\par
Schüler geben dazu lediglich ihre ID ein oder Scannen diese. Ist die eingegebene ID im System nicht vorhanden, wird zur erneuten Eingabe aufgefordert.\par
Für Veranstalter gibt es eine Schaltfläche, durch  welche die App in den Veranstaltermodus wechselt.
\item[AF020] Ein Veranstalter muss zum Anmelden seine E-Mail-Adresse eingeben. Diese wird an den Server übertragen, welcher bestätigen muss, dass die Adresse zu einem Veranstalter gehört. Der Server verfügt dazu über eine Liste von E-Mail-Adressen, die zu Veranstaltern gehören.\par
Ist die eingegebene E-Mail-Adresse in dieser Liste enthalten, erzeugt die App lokal ein Schlüsselpaar aus private und public key und sendet den public key an den Server. Der Server sendet an die E-Mail-Adresse automatisch eine E-Mail, die einen Bestätigungslink enthält. Der Veranstalter muss diesen aufrufen. Dadurch wird der public key freigegeben, so dass sich die App des Veranstalters mit diesem authentifizieren kann.\par 
Zu einem Veranstalter können mehrere Schlüssel (pro Gerät einer) gehören.
\item[AF030] Die Anmeldedaten werden gespeichert. Die App nutzt beim nächsten Neustart die gespeicherten Anmeldedaten.
\item[AF040]Ist der Benutzer angemeldet, hat er über das Menü die Funktion sich abzumelden. Dadurch werden die gespeicherten Anmeldedaten zurückgesetzt und der zur Anmeldung auffordernde Startbildschirm wird angezeigt. So kann die App von mehreren Benutzern auf dem selben Gerät verwendet werden.\par 
Die bei der Anmeldung eines Veranstalters erzeugten Schlüssel werden nicht gelöscht, sondern zusammen mit der E-Mail-Adresse gespeichert, so dass beim erneuten anmelden nur diese Adresse erneut angegeben werden muss.
\item[AF050] Für Veranstalter gibt es die Möglichkeit das für die Anmeldung verwendete Schlüsselpaar zu löschen. Dabei wird der Veranstalter abgemeldet.
\end{itemize}
\subsubsection*{Generelle Funktionen der App}
\begin{itemize}
%GF=Generelle Funktionen
\item[GF010] Die App kann bietet dem Nutzer die Möglichkeit, sich die Termine der nächsten Veranstaltungen anzeigen zu lassen. 
\item[GF020] Für den Fall einer Downtime des Servers, besitzt die App einen Offline Modus. Hier können entweder die zuletzt gültigen Daten eingesehen werden (von den Schülern), oder auch ID's eingescannt werden vom Veranstalter. Diese Daten werden zunächst lokal gespeichert und aktualisiert sobald eine Verbindung zu den Servern hergestellt wurde.
\end{itemize}
\subsubsection*{Punktevergabe}
\begin{itemize}
\item[PF010] Für eine Veranstaltung soll es auch einen Punktwert geben. Standardmäßig soll dieser ein Punkt pro Veranstaltung betragen.
\item[PF020] Die erreichten Punkte werden bei der persönlichen Statistik und in der Bestenliste angezeigt.
\item[PF030] Die Bestenliste für Schüler wird nach erreichten Punkten sortiert.
\end{itemize}
\subsubsection*{Funktionen für angemeldete Schüler}
\begin{itemize}
%SF steht für Schülerfunktion
\item[SF010] Ein angemeldeter Schüler kann seine persönliche Statistik (Anzahl besuchter Veranstaltungen) einsehen.
\item[SF020] Einem angemeldeten Schüler wird die eigene ID angezeigt. \par 
Dabei soll diese Anzeige bei einer Veranstaltung, anstatt des physischen Passes, eingescannt werden können.
\item[SF030] Ein angemeldeter Schüler kann eine Bestenliste, welche die Platzierung, ID und die Anzahl besuchter Veranstaltungen der besten 10 und des Schülers selbst anzeigt, einsehen.
\end{itemize}
\subsubsection*{Funktionen für Veranstalter}
\begin{itemize}
%VF=Veranstalterfunktionen
\item[VF010] Dem angemeldeten Veranstalter werden die eigenen Veranstaltungen angezeigt. Von diesen kann er eine Auswählen um den Scanvorgang zu starten.
\item[VF020] Der Scanvorgang bietet die Möglichkeit die ID von Teilnehmern bei einer Veranstaltung zu scannen. Die Anwesenheit der Teilnehmer für diese Veranstaltung wird dann automatisch in die Datenbank aufgenommen.\par 
Alternativ kann die ID manuell eingegeben werden.
\item[VF030] Dem angemeldetem Veranstalter werden bei seinen eigenen Veranstaltungen, die Anzahl der in der Datenbank gespeicherten Teilnehmer angezeigt.
\end{itemize}

\subsection{Kann-Ziele}
\subsubsection*{Anmeldefunktionen}
\begin{itemize}
\item[AF110] Um eine Belastung durch unautorisierte Anmeldeversuche zu verhindern, soll die App erkennen, wenn in ihr die gleiche E-Mail-Adresse mehrmals bei der Anmeldung als Veranstalter eingegeben wird und anstatt einen neuen Schlüssel zu erstellen, darauf hinweisen, dass bereits ein Bestätigungslink versendet wurde, solange dieser noch gültig ist.
\item[AF120] Die Bestätigungslinks sind nur 24h gültig.
\end{itemize}
\subsubsection*{Generelle Funktionen}
\begin{itemize}
\item[GF110] Die App kann personalisiert werden. (Z.B. Hintergrundfarbe) 
\end{itemize}

\subsubsection*{Funktionen für angemeldete Schüler}
\begin{itemize}
\item[SF110] Der angemeldete Schüler hat die Möglichkeit sich bereits im voraus unverbindlich für eine Veranstaltung anzumelden.
\item[SF120] Der angemeldete Schüler kann sich eine Historie der von ihm besuchten Veranstaltungen anzeigen lassen.
\item[SF130] Der angemeldete Schüler kann die Bestenliste nach verschiedenen Kriterien (z.B. Schule, Art der Veranstaltung) filtern.\footnote{Diese Erweiterung setzt voraus, dass genügt Daten, nach denen gefiltert werden kann, vorhanden sind.}
\item[SF140] Der angemeldete Schüler kann Veranstaltungen an denen er teilgenommen hat Bewerten.
\end{itemize}
\subsubsection*{Funktionen für Veranstalter}
\begin{itemize}
\item[VF110] Dem angemeldetem Veranstalter werden bei eigenen zukünftigen Veranstaltungen die Anzahl der Schüler, welche sich über die App im Voraus angemeldet haben, angezeigt.
\item[VF120] Der angemeldete Veranstalter kann sich eine Bestenliste seiner eigenen Veranstaltungen nach Anzahl der Teilnehmer oder Bewertung anzeigen lassen.
\end{itemize}

\section{Nichtfunktionale Anforderungen}
\subsection{Muss-Ziele}
\subsubsection*{Nutzung}
\begin{itemize}
\item[NF010] Die Oberfläche und Bedienung der App sollte einfach und intuitiv sein, damit auch nicht Technik affine Veranstalter sowie Schüler sie bedienen können.
\item[NF020] Der ID-Scanner soll zuverlässig und schnell arbeiten, so dass es Sinn macht, die ID nicht per Hand einzugeben.
\item[NF030] Ein Benutzer kann sich auf mehreren Geräten gleichzeitig anmelden.
\end{itemize}
\subsubsection*{Datenübertragung und Speicherung}
\begin{itemize}
\item[DF010] Die Datenübertragung der Veranstalter Daten soll verschlüsselt sein. Die App kennt dabei bereits den public key des Servers. Für die Erstellung des Schlüsselpaares der App siehe AF020.
\item[DF020] Es soll sichergestellt werden, dass gescannte Schüler auch bei nicht bestehender Internetverbindung als an der Veranstaltung teilgenommen eingetragen werden. In diesem Fall soll die Teilnehmerliste temporär lokal gespeichert werden und bei bestehender Internetverbindung übertragen werden.
\item[DF030] Es werden Kampagnen-Tags genutzt um das jeweilige Jahr zu bestimmen. D.h. der Zeitraum in dem Schüler Veranstaltungen besuchen und Punkte sammeln können, um ein Diplom zu bekommen, wird begrenzt und mit einem Tag versehen. Sodass nach Ablauf der Kampagne erhobene Daten zurückgesetzt oder ausgelagert werden können.
\end{itemize}
\section{Qualitätsmatrix nach ISO 25010}
\begin{tabular}{|l|l|}
\hline  Kriterien & Bedeutung \\
\hline Funktionale Software & hoch\\
\hline Effiziente Performance & niedrig\\
\hline Höchste Sicherheit & mittel\\
\hline Hohe Kompatibilität & niedrig\\
\hline Verlässliche Software & mittel\\
\hline Perfekte Nutzbarkeit & hoch\\
\hline Einfache Wartung & niedrig\\
\hline Leichte Portierbarkeit & niedrig\\
\hline
\end{tabular}

\section{Lieferumfang und Abnahmekriterien}
\subsection{Lieferumfang}
\paragraph{Front-End} Wir liefern eine Android-App, welche gleichermaßen von Schülern und Veranstaltern genutzt werden kann.
\paragraph{Back-End} Wir erstellen eine Datenbank im RDF-Format zum Speichern der Teilnehmer in Form von anonymen IDs, deren besuchten Veranstaltungen und eventuell ihrer Punktestände, auf die die App zugreifen kann. Für die Anbindung wird auf dem Server Fuseki als Web Service aus dem Apache Jena Framework verwendet. Außerdem werden auf dem Server Java Applikationen zum handlen diverser Anfragen benötigt. Zunächst wird diese Datenbank auf dem Praktikumsserver implementiert. Später soll jedoch die Möglichkeit bestehen, die Datenbank auf einen anderen Server zu verschieben.
\paragraph{Dokumentation} Für Schüler und Veranstalter soll es eine Webseite geben auf der die Bedienung der App erklärt wird. Für den Plattformbetreiben soll es neben der Entwurfsdokumentation eine PDF-Datei, die die Handhabung der Datenbank erläutert, geben. Des weiteren soll der Quellcode für die Weiterentwicklung durch andere Entwickler kommentiert und zusammen mit einer Javadoc vorliegen.
\subsection{Abnahmekriterien}
\paragraph{Front-End} Die Android-App muss auf jedem aktuellen Android Gerät installiert werden können und stabil laufen. Alle oben genannten Muss-Ziele sind zu erfüllen.
\paragraph{Back-End} Die Datenbank muss in der Lage sein, Daten sowohl über die App zu beziehen, als auch mit den Informationen der bereits bestehenden Primärplattformen zu arbeiten. 
\paragraph{Dokumentation} Die Dokumentation beinhaltet:
\begin{itemize}
\item Darstellung und Bedienung der Features aus Sicht des Endnutzers
\item Aspekte der Installation und Konfiguration für Systemadministratoren
\item Darstellung von Informationen und Komponenten der Software für Entwickler mit verschiedenen Detaillierungsgraden
\item Anforderungen und Entscheidungen während der Entwicklung für nachhaltige Kommunikation über den Kontext der Entwicklung
\end{itemize}

\section{Vorprojekt}
Als Vorprojekt soll eine Vorab Version der App konstruiert werden, die bereits einige der Muss-Ziele umgesetzt hat und zeigt dass diese funktionieren. \\
Als Erstes soll die Benutzeroberfläche bis dato größtenteils fertig sein und zeigen wie das Team die App designen möchte. \\
Des Weiteren soll die „Schüler-Seite“ mit Ihren wichtigsten Funktionen entwickelt worden sein. Die App soll eine ID erkennen können (diese muss eingetippt werden) und anschließend feststellen ob diese bereits in der Datenbank vorhanden ist[AF010]. Falls die ID unbekannt ist, so soll ein Fehler angezeigt werden. Andernfalls sollen der ID die jeweiligen Werte(z.B. Highscore) zugeordnet und angezeigt werden können[SF010/SF020]. Dazu wird eine Test-Datenbank entworfen welche einige manuell erstellte Datensätze enthält. Diese sollen testweise angezeigt werden, wenn die entsprechende ID erkannt wurde(bzw. hinterlegt ist). Und man sollte neben dem Punktestand  auch seinen Platz in einer Rangliste sehen können, verglichen mit anderen IDs [SF030]. \newline
Auf den „Veranstalter-Teil“ der App soll im Rahmen des Vorprojekts nicht weiter eingegangen werden.  


\section{Glossar}
\begin{sortedlist}
%Liste wird autoatisch alphabetisch nach dem Inhalt von [] sortiert.
\sortitem[ID]{(Identifikationsnummer) Im Zusammenhang dieses Projektes ist die Identifikationsnummer auf den Pässen der Schüler gemeint.}
\sortitem[Datenbank] {Eine Datenbank ist ein System zur effizienten, widerspruchsfreien und dauerhaften elektronischen Speicherung von Daten. Bei der Datenbank dieses Projekts handelt es sich um eine RDF Datenbank, Daten liegen im RDF Format vor.}
\sortitem[Plattformbetreiber] {Der Betreiber einer Plattform ist eine Organisation, die die für eine Verbreitung ihrer Angebote erforderliche Netzinfrastruktur entweder selber betreibt, durch Dritte betreiben lässt oder das offene Internet nutzt. Im Zusammenhang dieses Projektes ist der Plattformbetreiber die Stadt Leipzig, beziehungsweise die Verantwortlichen des Ferienpasses.}
\sortitem[RDF] {RDF steht für Resource Description Framework. Es wurde von der W3C als Standard zur Beschreibung von Metadaten konzipiert. Es gilt als ein grundlegender Baustein des Semantischen Webs. RDF ähnelt den klassischen Methoden zur Modellierung von Konzepten wie UML-Klassendiagramme und ER-Modell.}
\sortitem[Frontend] {Frontend beschreibt das auf dem Endgerät des Nutzers laufende Programm.}
\sortitem[Backend] {Backende beschreibt das auf dem Server laufende Programm, ein SPARQL Endpunkt (Fuseki, Apache Jena), sowie Java und Java Applikationen zum bearbeiten diverser Anfragen.}
\sortitem[Semantic Web] {Beim semantischen Web handelt es sich um eine Web-Technologie bei dem die Suchmaschinen Informationen zueinander in Beziehung setzen, sie eigenständig auswerten und aus ihnen eine Bedeutung entnehmen können.}
\sortitem[Secure Shell] {(kurz SSH) bezeichnet ein Netzwerkprotokoll, mit deren Hilfe man auf eine sichere Art und Weise eine verschlüsselte Netzwerkverbindung mit einem entfernten Gerät herstellen kann.}
\sortitem[private key]{Privater Schlüssel -- Teil eines Schlüsselpaares das nur dem Besitzer bekannt ist und mit dem durch den dazugehörigen public key verschlüsselte Daten entschlüsselt werden können.}
\sortitem[public key] {Öffentlicher Schlüssel -- Teil eines Schlüsselpaares mit dem Nachrichten an den Besitzer dieses Schlüssels verschlüsselt werden können.}
\sortitem[Schüler] {Der Schüler nimmt am Ferienpass teil. Er soll die App leicht bedienen können.}
\sortitem[Veranstalter] {Der Veranstalter organisiert Freizeitaktivitäten für die Schüler. Er soll mit der App den Code des Schülers einscannen.}
\end{sortedlist}

\end{document}