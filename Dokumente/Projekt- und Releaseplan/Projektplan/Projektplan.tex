\documentclass[11pt,a4paper]{scrartcl}
\usepackage[utf8]{inputenc}
\usepackage[T1]{fontenc}
\usepackage[ngerman]{babel}
\usepackage[top=2.5cm, bottom=3cm]{geometry}
\usepackage{scrpage2}

%\usepackage{cite}

\pagestyle{scrheadings}
\clearscrheadfoot
\ihead{\today \\
Marcel Herhold
}  
\ohead{hg17b}
\cfoot{\pagemark}

\renewcommand*{\thesection}{\arabic{section}.}

\begin{document}
\subject{\vspace{-2cm}}
\title{\vspace{-0.5em}Projektplan}
\subtitle{\vspace{1ex}hg17b - Android App für Weiterbildungsmanagement}
\author{Marcel Herhold\vspace{-0.5em}}%autor
\date{\vspace{-0.5em}\today}
\maketitle

%\tableofcontents
%\newpage
\section{Arbeitspaket 1}
\subsection*{Benutzeroberfläche Schüler (10\%)}

Die Schülerseite der App soll nach dem einscannen oder eintippen einer ID anzeigen an welchen Events der Schuler teilgenommen hat , an wieviele er insgesammt Teilgenommen hat, seinen Rang in der Bestenliste sowie die Top 10 der Rangliste. Die Schüler sollen auch sehen können an welchen Events sie bereit Teilgenommen haben. Als kann Ziele soll zusätzlich die möglichkeit bestehen sind für Events vorab anzumelden sowie eine Bewertungfunktion für besuchte Events.

\section{Arbeitspaket 2}
\subsection*{Benutzeroberfläche Veranstalter  (10\%)}

Die Veranstalterseite soll über eine Anmeldung mittels SSH-Key geschehen , darin soll es ein Menü geben welches dem Veranstalter eine Liste seiner eigenen Events gibt sowie die Möglichkeit des Veranstalters eingescannte IDs diesen Events zuzuordnen um anzuzeigen wer daran teilgenommen hat. Die Veranstalter sollen außerdem sehen können wieviele Schüler an ihren Events teilgenommen haben.  Als evtl Zusatzleistungen soll der Veranstalter sehen können wieviele Schüler sich für ihre kommenden Event angemeldet haben und außerdem die Bewertung ihrer Events durch die Schüler.

\section{Arbeitspaket 3}
\subsection*{Server (20\%)}

In diesem Arbeitspaket soll alles grundlegende des Servers eingerichtet werden um in einem spätern Arbeistpaket dann den RDF-Store sowie den E-mail Service für die Bestätigungsmails. Dazu muss der Server eingerichtet werden und die Schnittstellen für die Kommunikation mit der App vorbereitet werden.

\section{Arbeitspaket 4}
\subsection*{RDF-Store (10\%)}

Der RDF-Store der in diesem Arbeitspaket entsteht soll die Daten der Schüler speichern dazu gehören die IDs der Schüler , an welchen Events sie Teilgenommen haben , aus welchem Jahr sie sind , ihre bisherigen Punkte und somit auch ihren Platz in der Bestenliste und evtl auch an welchen Events sie noch Teilnehmen wollen. Außerdem soll hier auch alle Events und ihre Veranstalter gespeichter werden um schnell darauf zugreifen zu können sowie um die Punkte die diese Veranstaltungen wert sind hinterlegt zu haben.

\section{Arbeitspaket 5}
\subsection*{Kommunikation mit dem Server (30\%)}

Im vierten Arbeitspaket soll die Schnittstelle zwischen dem Server und der App hergestellt werden. Der Server muss hierbei zwischen einer Anfrage der Schülerseite und der Veranstalterseite unterscheiden können um die richtigen Daten herrauszugeben bzw zu verändern. Damit die App gegen Downtimes des Servers geschützt ist sollen hier auch bei der App Daten zwischengespeichter werden die dann wenn der Server wieder funktioniert an diesen gesendet werden können.

\section{Arbeitspaket 6}
\subsection*{Coderscanner (5\%)}

Im Arbeitspaket 6 soll der Codescanner realisiert werden. Mit diesem soll es dem Veranstaltern möglich sein schnell und bequem alle zu ihren Veranstaltungen kommenden Schüler/innen im System einzutragen (die ID von ihnen zu scannen).

\section{Arbeitspaket 7}
\subsection*{Anmeldung (45\%)}

  
  
\section{Arbeitspaket 8}
\subsection*{Schlüsselgeneration (5\%)}

  
  
\section{Arbeistpaket 9}
\subsection*{Speicherung der Schlüssel des Veranstalters (5\%)} 



\section{Arbeistpaket 10}
\subsection*{E-Mail-Service für Bestätiungsmails (10\%)}

      

\end{document}