\documentclass[11pt,a4paper]{scrartcl}
\usepackage[utf8]{inputenc}
\usepackage[T1]{fontenc}
\usepackage[ngerman]{babel}
\usepackage[top=2.5cm, bottom=3cm]{geometry}
\usepackage{scrpage2}

\usepackage{cite}

\pagestyle{scrheadings}
\clearscrheadfoot
\ihead{\today \\
Christopher Pfeiffer
}  
\ohead{hg17b}
\cfoot{\pagemark}

\renewcommand*{\thesection}{\arabic{section}.}

\begin{document}
\subject{\vspace{-2cm}}
\title{\vspace{-0.5em}Releaseplan}
\subtitle{\vspace{1ex}hg17b - Android App für Weiterbildungsmanagement}
\author{Christopher Pfeiffer\vspace{-0.5em}}%autor
\date{\vspace{-0.5em}\today}
\maketitle

%\tableofcontents
%\newpage
\section{Release (Vorprojekt)}
\paragraph{Termin:} 29.01.2018
\subsection*{Beschreibung der umgesetzten Funktionalität}
Die App soll Testdaten von Schülern vom Server abfragen und anzeigen können. Dazu soll der Server inklusive RDF-Store aufgesetzt sein und die Oberfläche für die Schülerseite der App fertig sein. Außerdem muss die App mit dem Server kommunizieren können. Die Eingabe der Schüler-ID soll erst einmal über ein normales Eingabefeld geschehen.
\paragraph{Dazugehöriges Arbeitspaket aus Projektplan:}

\section{Release}%2
\paragraph{Termin:} 16.02.2018
\subsection*{Beschreibung der umgesetzten Funktionalität}
Die Veranstalterseite der App soll (noch ohne sichere Anmeldefunktion) fertiggestellt sein und Informationen für einen Veranstalter anzeigen können. Dazu soll der Zugriff auf die Veranstaltungsdatenbank integriert werden.\\
Als zusätzliches Kann-Ziel sollen Einstellungen zur Personalisierung der App hinzugefügt werden.
\paragraph{Dazugehöriges Arbeitspaket aus Projektplan:}
 
\section{Release}%3
\paragraph{Termin:} 12.03.2018
\subsection*{Beschreibung der umgesetzten Funktionalität}
Anmeldefunktionen für die Veranstalter sollen hinzugefügt werden. Dies beinhaltet die automatische Erstellung von Schlüsselpaaren durch die App, sowie die Möglichkeit zur Verifizierung mittels Bestätigungs-E-Mail. Dazu muss der Server zusätzlich E-Mails versenden können.
\paragraph{Dazugehöriges Arbeitspaket aus Projektplan:}

\section{Release}%4
\paragraph{Termin:} 26.03.2018
\subsection*{Beschreibung der umgesetzten Funktionalität}
Der Scanner zum scannen von IDs soll eingefügt werden. Außerdem soll die App als Gesamtheit auf Stabilität und Erfüllung der notwendigen Funktionen getestet werden.\\
Als zusätzliches Kann-Ziel sollen sich Schüler im Voraus für Veranstaltungen anmelden können und dies den Veranstaltern angezeigt werden.
\paragraph{Dazugehöriges Arbeitspaket aus Projektplan:} 

\section{Release}%5
\paragraph{Termin:} 05.04.2018
\subsection*{Beschreibung der umgesetzten Funktionalität}
Die Dokumentation der App soll finalisiert werden und die im Bereich des 4. Releases gefundenen Fehler sollen behoben werden. Die App soll bereit für den realen Einsatz sein.
\paragraph{Dazugehöriges Arbeitspaket aus Projektplan:} 

\end{document}